\documentclass[12pt]{article}
\usepackage[utf8]{inputenc}
\usepackage[letterpaper]{geometry}
\usepackage{amsmath}
\usepackage{amsfonts}

% typesetting issues
\setlength{\textwidth}{5.5in}
\setlength{\textheight}{9.5in}
\setlength{\oddsidemargin}{0.5\paperwidth}
\addtolength{\oddsidemargin}{-1.0in} % latex oddity
\addtolength{\oddsidemargin}{-0.5\textwidth}
\setlength{\topmargin}{0.5\paperheight}
\addtolength{\topmargin}{-1.0in} % latex oddity
\addtolength{\topmargin}{-0.5\headheight}
\addtolength{\topmargin}{-0.5\headsep}
\addtolength{\topmargin}{-0.5\textheight}
\markboth{}{Hogg \& Villar / Coordinate-free machine learning}
\pagestyle{myheadings}
\linespread{1.08}
\frenchspacing\raggedbottom\sloppy\sloppypar

% math issues -- Hogg is trying to go all semantic here.
\newcommand{\inv}{^{-1}}
\newcommand{\T}{^\top}
\newcommand{\R}{{\mathbb R}}
\newcommand{\surf}{{\mathrm{s}}}

\title{\bfseries%
  Towards truly coordinate-free\\ machine learning}
\author{%
  David W. Hogg, 
  Soledad Villar, and
  others}
\date{2022 June}

\begin{document}

\maketitle\thispagestyle{empty}

\paragraph{Abstract:}
There has been abundant work on making machine-learning methods exactly (or approximately) equivariant to group actions.
Equivariant functions obey relations like $f(g\cdot x) = g\cdot f(x)$; that is, if the inputs $x$ are transformed by group element $g$, then the outputs $f(x)$ are similarly transformed.
Equivariance is the correct goal when the rules of the game have symmetries, or when the input--output relationship is desired to be invariant to group operations.
There have been successful implementations for translation (and subgroups), rotation (and subgroups), reflection, permutation, boost, and many others.
The laws of physics contain a much deeper symmetry than any of these, however:
Even in situations in which the physical law is not symmetric, it must be possible to describe the law in a coordinate-free way.
These situations are common; for example the physics near the surface of the Earth is very strongly oriented:
Free objects fall in the down direction, usually.
And yet the laws can be expressed in a perfectly coordinate-free way by making use of the local gravitational acceleration vector.
Coordinate freedom looks a lot like equivariance, but it only takes the precise form $f(g\cdot x)=g\cdot f(x)$ when certain (hopefully learnable) geometric constants are included in the input $x$.
Here we explore these ideas, with toy asymmetric physics systems used as examples.
We conjecture that coordinate-free methods will improve out-of-sample generalization even in cases in which the functions being learned are not themselves equivariant.

\section{Introduction}

Hello World!

An illustrative example might be the following:
In the study of equivariant machine learning, the springy double pendulum is often used as a toy problem for demonstrations \cite{Finzi}.
The final conditions (position and velocity after elapsed time $T$) are related to the initial conditions (position and velocity at the initial time).
The vectors are 3-dimensional, but the is problem is naturally $O(2)$-equivariant (equivariant with respect to rotations and reflections in the 2-d plane), since there is a gravitational vector that points downwards, breaking the 3-dimensional symmetry.
However, if the gravitational vector is included as a feature in the problem, the problem becomes $O(3)$-equivariant; we showed that including this input and obtaining the higher symmetry is better for predictive accuracy both inside and outside of the support of the training set \cite{Yao}.
The coordinate freedom of the problem is a kind of $O(3)$ equivariance, but this equivariance only appears when the geometric constants (the gravitational force vector in this case) are included as inputs.

\paragraph{Our contributions:}

\paragraph{Prior work:}

\section{Problem setup and definitions}

The problems we are interested in involve learning a function from data, as with machine learning and regression.
The function we will learn takes scalar, vector, and tensor inputs, and delivers scalar, vector, and tensor outputs.
Scalars, vectors, and tensors are defined by the transformation properties of their components under changes of coordinates.
If $G$ is the group of rotations, reflections, and dilations (is there such a group?):
\begin{itemize}
  \item We say that $a\in\R$ is a scalar if, when the coordinate system is modified according to group element $g\in G$, $a$ does not change.
  \item We say that $a\in\R^d$ is a vector if, when the coordinate system is modified according to group element $g\in G$, the components of $a$ change to $g\inv\cdot a$.
  \item We say that $a\in(\R^d)^k$ is a $k$-tensor if, when the cordinate system is modified according to group element $g\in G$, the components of $a$ change to \ldots
\end{itemize}
[Obviously we have to mathify the above and be clear about what groups are in play.]

It would be good to talk about translations too---Hogg believes that all of the above are invariant to translations.

[Now discuss the inputs to the function. Most known, a few unknown.]

[Now define coordinate-freedom. What is a coordinate-free expression? I think it is something that is equivariant once all the unknown relevant quantities are included? But that seems wishy-washy. And I'd love a description that doesn't reduce to equivariance exactly.]

[Should we talk about units too? Is that a kind of coordinate freedom too?]

We're going to think about natural-science applications, where there is no equivariance---the relevant functions we are trying to learn are not invariant or equivariant to translations and rotations (say)---but where we are exceedingly confident that the functions must be expressible in coordinate-free form.

\section{Experiments}

\paragraph{Bouncing ball:}
Our first experimental setup is the dynamics of a particle, thrown or dropped near a uniformly moving planar elastic surface.
The problem will be to predict the momentum and position $(p,q)$ of the particle at an elapsed time $T$ after an initial time at which the momentum and position were last known.

Written in terms of invariant scalar products, the non-relativistic Hamiltonian $H(p,q)$ is
\begin{align}
    H(p,q) &= K + V \\
    K(p) &= \frac{1}{2\,m}\,(p - m\,v_\surf)\T (p - m\,v_\surf) \\
    V(q) &= m\,g\T (q_\surf - q) + \frac{1}{2}\,k\,x^2\,I_{x<0} \\
    x &:= \frac{q_\surf\T}{\sqrt{q_\surf\T q_\surf}}\,(q_\surf - q) \\
    I_{x<0} &:= \left\{\begin{array}{cl} 1 & x < 0 \\ 0 & x \geq 0\end{array}\right. ~,
\end{align}
where $p,q$ are the vector momentum and position (in some coordinate system), $m$ is the scalar particle mass, $v_\surf$ is the vector velocity of the planar (equilibrium) surface, $g$ is the vector acceleration due to gravity, $k$ is a scalar effective spring constant for the surface, $q_\surf$ is the vector pointing from the coordinate origin to the closest point on the planar (equilibrium) surface (this is a function of time if $v_\surf$ is non-zero), $I_{x<0}$ is a scalar indicator function, and $x$ is a scalar length which is the (signed) perpendicular distance between the particle and the surface.
Some notes about this Hamiltonian:
\begin{itemize}
  \item The vectors are implicitly column vectors, such that products like $g\T (q_\surf - q)$ are scalars.
  \item This expression for the Hamiltonian is explicitly constructed from classical-physics invariants, and these invariants are invariant to rotations, reflections, translations, and (Galilean) boosts.
  \item This Hamiltonian is not time-invariant: If $v_\surf$ is non-zero, then $q_\surf$ varies with time. Thus in general the total mechanical energy will not be invariant.
\end{itemize}

\paragraph{Springy double pendulum:}
Hello World!

\section{Discussion}

Hello World!

\end{document}
