\documentclass[12pt]{article}
\usepackage[utf8]{inputenc}
\usepackage[letterpaper]{geometry}
\usepackage{amsmath}
\usepackage{amsfonts}

\title{\bfseries%
  Towards truly coordinate-free machine learning}
\author{David W. Hogg}
\author{Soledad Villar}
\author{others}
\date{2022 June}

\frenchspacing\raggedbottom\sloppy\sloppypar
\begin{document}

\maketitle\thispagestyle{empty}

\paragraph{Abstract:}
There has been abundant work on making machine-learning methods exactly (or approximately) equivariant to group actions.
That work structures functions such that $f(g\cdot x) = g\cdot f(x)$; that is, if the inputs $x$ are transformed by group element $g$, then the outputs $f(x)$ are similarly transformed.
Equivariance is the correct goal when the rules of the game have symmetries, or when the input--output relationship is desired to be invariant to group operations.
There have been successful implementations for translation (and subgroups), rotation (and subgroups), reflection, permutation, boost, and many others.
The laws of physics contain a much deeper symmetry than any of these, however:
Even in situations in which the physical law is not symmetric, it must be possible to describe the law in a coordinate-free way.
These situations are common; for example the physics near the surface of the Earth is very strongly oriented:
Free objects fall in the down direction, usually.
And yet the laws can be expressed in a perfectly coordinate-free way by making use of the local gravitational acceleration vector.
Coordinate freedom looks a lot like equivariance, but it only takes the precise form $f(g\cdot x)=g\cdot f(x)$ when certain geometric constants are included as inputs or learned.
Here we explore these ideas, with toy asymmetric physics systems used as examples.
We conjecture that coordinate-free methods will improve out-of-sample generalization even in cases in which the functions being learned are not equivariant.

\end{document}
